%Compile using xelatex
\documentclass[12pt]{article}
\usepackage{graphicx}
\usepackage{xltxtra,xgreek,fontspec} 
\usepackage{color}
\setmainfont{FreeSerif} % main font
\usepackage{setspace}
\usepackage[export]{adjustbox}[2011/08/13]
\usepackage[usenames,dvipsnames,svgnames,table]{xcolor}
\usepackage[textwidth=20cm, textheight=25cm]{geometry}
\usepackage[greek]{datetime2}
\usepackage{caption}
\captionsetup[figure]{name=Εικόνα, labelfont=it} % custom name for image description 
\usepackage{float}
\usepackage{fancybox}
\usepackage{hyperref} % link customization
\hypersetup{
	colorlinks,
	linkcolor=black,
	citecolor={blue!50!black},
	urlcolor={blue!80!black}
}
\usepackage{tikz}
\usetikzlibrary{shadows}

\newcommand*\keystroke[1]{%
	\tikz[baseline=(key.base)]
	\node[%
	draw,
	fill=white,
	drop shadow={shadow xshift=0.25ex,shadow yshift=-0.25ex,fill=black,opacity=0.75},
	rectangle,
	rounded corners=2pt,
	inner sep=2pt,
	line width=0.5pt,
	font=\scriptsize\ttfamily
	](key) {#1\strut}
	;
}

\newfontfamily\myfont{GFS Didot} %custom font for university and dept. name
%\usepackage{fancyhdr}
%\pagestyle{fancy}
%\fancyhf{}
%\rhead{\footnotesize Lab 02} % right header
%\lhead{\footnotesize Λύσεις} %left header
%%\chead{\includegraphics[scale=0.25]{C:/uowm.png}}
%\cfoot{\noindent\makebox[\linewidth]{\rule{0.75cm}{0.4pt}}\\\thepage} %custom page number format
\graphicspath{{./images/}}

\usepackage[flushmargin, hang, bottom]{footmisc}
\addtolength{\footnotesep}{2.5mm} % change to 1mm

\usepackage{infobox}

\begin{document}
	
\noindent\textbf{\Large Λύσεις εργαστηρίου 2b\\
«Διευθέτηση μεταγωγέα»}

\hrulefill


\subsection*{Παραμετροποίηση δρομολογητή}
\begin{tabular}{m{12cm}m{7cm}}
	\texttt{Router> enable} & Ενεργοποίση προνομιακής κατάστασης\\[0.25cm]
	\texttt{Router\# conf term} & Ενεργοποίηση κατάστασης ρυθμίσεων μέσω τερματικού\\[0.25cm]
	\texttt{Router(conf)\# interface GigabitEthernet0/1} & Είσοδος σε κατάσταση ρυθμίσεων για την διεπαφή \texttt{GE0/1}\\[0.25cm]
	\texttt{Router(conf-if)\# ip address 192.168.1.1 255.255.255.0} & Ανάθεση της IP \texttt{192.168.1.1} και μάσκας \texttt{/24} στην συγκεκριμένη διεπαφή\\[0.25cm]
	\texttt{Router(conf-if)\# no shutdown} & Ενεργοποίηση της διεπαφής\\[0.25cm]
	\texttt{Router(conf-if)\# exit} & Έξοδος από την κατάσταση ρυθμίσεων της διεπαφής\\[0.25cm]
\end{tabular}

\subsection*{Παραμετροποίηση μεταγωγέα}
\begin{tabular}{m{9cm}m{10cm}}
	\texttt{Switch> enable} &Ενεργοποίση προνομιακής κατάστασης \\[0.25cm]
	\texttt{Switch\# conf term} & Ενεργοποίηση κατάστασης ρυθμίσεων μέσω τερματικού\\[0.25cm]
	\texttt{Switch(conf)\# interface interface Gi1/0/1} & Είσοδος σε κατάσταση ρυθμίσεων για την διεπαφή \texttt{GE1/0/1}\\[0.25cm]
	\texttt{Switch(conf-if)\# no shutdown} & Ενεργοποίηση της διεπαφής\\[0.25cm]
	\texttt{Switch(conf-if)\# exit} &Έξοδος από την κατάσταση ρυθμίσεων της διεπαφής\\[0.25cm]
	\texttt{Switch(conf)\# interface Gi1/0/2} & Είσοδος σε κατάσταση ρυθμίσεων για την διεπαφή \texttt{GE1/0/2} \\[0.25cm]
	\texttt{Switch(conf-if)\# no shutdown} & Ενεργοποίηση της διεπαφής\\[0.25cm]
	\texttt{Switch(conf-if)\# exit} & Έξοδος από την κατάσταση ρυθμίσεων της διεπαφής\\[0.25cm]
	\texttt{Switch(conf)\# interface Gi1/0/4} & Είσοδος σε κατάσταση ρυθμίσεων για την διεπαφή \texttt{GE1/0/4} \\[0.25cm]
	\texttt{Switch(conf-if)\# no shutdown} & Ενεργοποίηση της διεπαφής\\[0.25cm]
	\texttt{Switch(conf-if)\# exit} &Έ ξοδος από την κατάσταση ρυθμίσεων της διεπαφής
\end{tabular}

\subsection*{Άσκηση}
\begin{tabular}{m{12cm}m{7cm}}
	\texttt{Switch> enable} &Ενεργοποίση προνομιακής κατάστασης \\[0.25cm]
	\texttt{Switch\# conf term} & Ενεργοποίηση κατάστασης ρυθμίσεων μέσω τερματικού\\[0.25cm]
	\texttt{Switch(conf)\# interface Gi1/0/2} & Εισέλθετε στην κατάσταση ρυθμίσεων της διεπαφής \texttt{GigabitEthernet1/0/1}\\[0.125cm]
	\texttt{Switch(conf-if)\# switchport mode access} & Ορίστε την κατάσταση λειτουργίας της διεπαφής σε access.\\[0.25cm]
	\texttt{Switch(conf-if)\# switchport port-security} & Ενεργοποιείστε την ασφάλεια θύρας.\\[0.25cm]
	\texttt{Switch(conf-if)\# switchport port-security mac-address ΧΧΧΧ.ΧΧΧΧ.ΧΧΧΧ} & Ορίστε την διεύθυνση MAC του PC2 ως την μοναδική διεύθυνση MAC που επιτρέπεται να συνδεθεί στην διεπαφή.\\[0.25cm]
	\texttt{Switch(conf-if)\# switchport port-security violation restrict} & H παραβίαση του κανόνα θα οδηγήσει σε παρεμπόδιση μετάδοσης δεδομένων για την συγκεκριμένη διεπαφή.\\[0.25cm]
	\texttt{Switch(conf-if)\# exit} &Έξοδος από την κατάσταση ρυθμίσεων της διεπαφής
\end{tabular}

\end{document}